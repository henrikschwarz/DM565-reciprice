\documentclass{article}

\usepackage[T1]{fontenc} 
\usepackage[utf8]{inputenc} 
\usepackage[UKenglish]{babel}
\usepackage{graphicx} 
\usepackage{listings} 
\usepackage[hidelinks]{hyperref}
\usepackage{enumitem} 
\usepackage{amsmath} 
\usepackage{tikz} 
\usepackage{float}
\usepackage{verbatim} 
\title{DM565 Innovation Project} 
\author{authors} 
\date{\today}


\begin{document}

\maketitle

\newpage

\tableofcontents

\newpage

\section{Introduction}
This report concerns the innovation project in the Formal Languages and Data Processing
course. The task of this project is to find an idea for a product involving some type of
open data, and then  evaluate the this idea as a
basis for a startup in a structured manner with the business model canvas.
Finally we need to construct a prototype showing the idea in practice. 
\section{Idea Description}
\begin{comment}
 we need to focus our idea, as discussed in the midwayseminar. The best is to have a core
idea that we focus on. 
\end{comment}



\section{Idea Evaluation}

For evaluating our idea the business model canvas will be used. The business model canvas
is a template for systematically documenting the business model of an existing business or
prospect startup. The business model canvas focuses on four different key areas of a
building a sustainable business: \begin{enumerate}[itemsep=0pt]
  \item Infrastructure
  \item Offering
  \item Customers
  \item Finances
\end{enumerate}

\subsection{Infrastructure}

\subsection{Offering}

\subsection{Customers}

 \subsection{Finances}
\section{Prototyping}

\end{document}
